\documentclass[10pt]{article}
\usepackage{amsmath,amssymb,bm,color, amsthm}
\usepackage[a4paper,margin=1.2in]{geometry}
\usepackage{indentfirst}
\usepackage{setspace}
\linespread{1.3}

\title{\textbf{Author Correction for}\\[3pt]
\emph{TensorProjection Layer: A Tensor-Based Dimension Reduction Method in Deep Neural Networks}}
\author{Toshinari Morimoto \and Su-Yun Huang}
\date{October 2025}

\begin{document}
\maketitle

\section*{Correction for Equation (11)}

The authors found that in the published paper, the derivative
\[
\frac{\partial\,\mathrm{vec}(G_k)}{\partial\,\mathrm{vec}(M_k)}, ~{\rm where}~ G_k = M_k^{-1/2},
\]
was not correct. 
The corrected expression is given by
\[
\boxed{
\displaystyle
\frac{\partial\,\mathrm{vec}(G_k)}{\partial\,\mathrm{vec}(M_k)^{\!\top}}
= - (M_k^{-1/2} \otimes M_k^{-1/2})
\bigl(I_{q_k} \otimes M_k^{1/2} + M_k^{1/2} \otimes I_{q_k}\bigr)^{-1}.
}
\tag{1}
\]
This correction does \textbf{not} affect the numerical results,
since all experiments used automatic differentiation
and did not rely on the analytical backpropagation formula.

\begin{proof}
Let $H_k = M_k^{1/2}$,
where each $G_k, H_k, M_k \in \mathbb{R}^{q_k \times q_k}$ and $M_k$ is symmetric strictly positive definite.
We aim to find the Jacobian via the chain rule:
\[
\frac{\partial\,\mathrm{vec}(G_k)}{\partial\,\mathrm{vec}(M_k)^{\!\top}}
=
\frac{\partial\,\mathrm{vec}(G_k)}{\partial\,\mathrm{vec}(H_k)^{\!\top}}
\frac{\partial\,\mathrm{vec}(H_k)}{\partial\,\mathrm{vec}(M_k)^{\!\top}}.
\]

\noindent
{\bf Step 1: Relation between $\bm{G_k}$ and $\bm{H_k}$.}
From $G_k H_k = I_{q_k}$, we have
\[
G_k(\Delta H_k) + (\Delta G_k)H_k = O, 
\]
where $O$ is the zero matrix and $\Delta$ denotes the differential. Then
\[
\Delta G_k = -\,G_k (\Delta H_k) H_k^{-1} = -\,G_k (\Delta H_k) G_k.
\]
Vectorizing both sides gives
\[
\mathrm{vec}(\Delta G_k)
= - (H_k^\top \otimes G_k)\,\mathrm{vec}(\Delta H_k),
\]
so that
\[
\frac{\partial\,\mathrm{vec}(G_k)}{\partial\,\mathrm{vec}(H_k)^{\!\top}}
= - (H_k^\top \otimes G_k)
= - (H_k^{-1} \otimes H_k^{-1})
= - (M_k^{-1/2} \otimes M_k^{-1/2}).
\tag{2}
\]

\noindent
{\bf Step 2: Relation between $\bm{H_k}$ and $\bm{M_k}$.}
Since $H_k^2 = M_k$, we have
\[
\Delta M_k = H_k (\Delta H_k) + (\Delta H_k) H_k.
\]
Using the identity $\mathrm{vec}(A X B) = (B^\top \otimes A)\,\mathrm{vec}(X)$,
\begin{align*}
\mathrm{vec}(\Delta M_k)
&= (I_{q_k} \otimes H_k)\,\mathrm{vec}(\Delta H_k)
  + (H_k \otimes I_{q_k})\,\mathrm{vec}(\Delta H_k) \\[1mm]
&= \bigl(I_{q_k} \otimes H_k + H_k \otimes I_{q_k}\bigr)\,\mathrm{vec}(\Delta H_k).
\end{align*}
Hence,
\[
\frac{\partial\,\mathrm{vec}(H_k)}{\partial\,\mathrm{vec}(M_k)^{\!\top}}
= \bigl(I_{q_k} \otimes H_k + H_k \otimes I_{q_k}\bigr)^{-1}.
\tag{3}
\]


\noindent{\bf Step 3: Chain rule.}
Combining (2) and (3),
\begin{align*}
\frac{\partial\,\mathrm{vec}(G_k)}{\partial\,\mathrm{vec}(M_k)^{\!\top}}
&=
\frac{\partial\,\mathrm{vec}(G_k)}{\partial\,\mathrm{vec}(H_k)^{\!\top}}
\frac{\partial\,\mathrm{vec}(H_k)}{\partial\,\mathrm{vec}(M_k)^{\!\top}} \\[1mm]
&=
- (H_k^{-1} \otimes H_k^{-1})
\bigl(I_{q_k} \otimes H_k + H_k \otimes I_{q_k}\bigr)^{-1}.
\end{align*}
\end{proof}

\noindent
{\bf Invertibility Remark.}
The matrix 
\[
I_{q_k} \otimes M_k^{1/2} + M_k^{1/2} \otimes I_{q_k}
\]
is invertible if and only if $M_k$ is strictly positive definite,
since its eigenvalues are $\{\sqrt{\lambda_i}+\sqrt{\lambda_j}\}_{i,j}$,
where $\lambda_i$ are the eigenvalues of $M_k$.

\end{document}
